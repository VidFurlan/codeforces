\documentclass{article}
\usepackage{titlesec}

\title{Competitive Programming Notes}
\author{Vid Furlan}
\date{2024}

\titleformat{\section}[hang]{\normalfont\Large\bfseries}{\thesection.}{0pt}{}
\titleformat{\subsection}[hang]{\normalfont\large\bfseries}{\thesubsection.}{0pt}{}


\begin{document}

\begin{titlepage}
    \begin{center}
        \vspace*{4cm}
        
        \Huge
        \textbf{Competitive Programming Notebook}
        
        \vspace{0.5cm}
        \LARGE
        - Contest Notes -
        
        \vspace{1.5cm}
        
        2024 Revision 0.0\\
        \vspace{0.5cm}
        Vid Furlan / vidfurlan
    \end{center}
\end{titlepage}

\renewcommand{\contentsname}{\Large Index}
\tableofcontents
\addcontentsline{toc}{section}{Index}

\newpage

\section{Introduction}
This is the introduction section. You can write an overview of your competitive programming notes here.

\newpage

\section{STL - Standard Template Library}
\subsection{Vector}
Vector is a dynamic array. It is similar to an array, but with the ability to resize itself automatically when it grows or shrinks.

\subsubsection{General Syntax:}
\begin{verbatim}
// 1D vector
vector<data_type> v1(size, value);

// 2D vector
vector<vector<data_type>> v2(row, vector<data_type>(col, value));

// Insertion O(1)
v.push_back(value);

// Deletion O(1)
v.pop_back();
\end{verbatim}

\subsubsection{Useful Functions:}
\begin{verbatim}
// Sort vector in ascending order O(nlogn)
sort(v.begin(), v.end());

// Sort vector in descending order O(nlogn)
sort(v.rbegin(), v.rend());

// Reverse vector O(n)
reverse(v.begin(), v.end());
\end{verbatim}

\subsection{Stack}
\subsection{Queue}
\subsection{Deque}
\subsection{Priority Queue}
\subsection{Set and Multiset}
\subsection{Map and Multimap}

\end{document}

